\documentclass[fontsize=12pt, paper=a4, twoside=false, numbers=noenddot, toc=listof, toc=bibliography]{scrreprt}

% Allgemeines
\usepackage[automark]{scrpage2} % Kopf- und Fußzeilen
\usepackage{amsmath,marvosym} % Mathematik
\usepackage[T1]{fontenc} % Ligaturen, richtige Umlaute im PDF
\usepackage[utf8]{inputenc}% UTF8-Kodierung für Umlaute usw

% Schriften
\usepackage{mathpazo} % Palatino für Mathemodus
\usepackage{setspace} % Zeilenabstand
\onehalfspacing % 1,5 Zeilen

% Schriften-Größen
\setkomafont{chapter}{\Huge\rmfamily} % Überschrift der Ebene
\setkomafont{section}{\Large\rmfamily}
\setkomafont{subsection}{\large\rmfamily}
\setkomafont{subsubsection}{\large\rmfamily}
\setkomafont{chapterentry}{\large\rmfamily} % Überschrift der Ebene in Inhaltsverzeichnis
\setkomafont{descriptionlabel}{\bfseries\rmfamily} % für description Umgebungen
\setkomafont{captionlabel}{\small\bfseries}
\setkomafont{caption}{\small}

%Zitate (jurabib)
\usepackage{jurabib}

\jurabibsetup{
	commabeforerest,
	ibidem=strict,
	citefull=all, %first als Standard
	see,
	titleformat={colonsep,all},
	authorformat=year
}
\renewcommand*{\jbauthorfont}{\textsc}
\renewcommand*{\biblnfont}{\scshape\textbf}
\renewcommand*{\bibfnfont}{\normalfont\textbf}

% Sprache: Deutsch
\usepackage[ngerman,pdfauthor={Max Hesselbarth},  pdfauthor={Max Hesselbarth}, pdftitle={T3300_6140175}]{hyperref}

% PDF
\usepackage[final]{microtype} % mikrotypographische Optimierungen
\usepackage{url}
\usepackage{pdflscape} % einzelne Seiten drehen können

% Tabellen
\usepackage{multirow} % Tabellen-Zellen über mehrere Zeilen
\usepackage{multicol} % mehre Spalten auf eine Seite
\usepackage{tabularx} % Für Tabellen mit vorgegeben Größen
\usepackage{longtable} % Tabellen über mehrere Seiten
\usepackage{array}

% Umlaute
\usepackage{bibgerm} % Umlaute in BibTeX

% Tabellen
\usepackage{multirow} % Tabellen-Zellen über mehrere Zeilen
\usepackage{multicol} % mehre Spalten auf eine Seite
\usepackage{tabularx} % Für Tabellen mit vorgegeben Größen
\usepackage{array}
\usepackage{float}

% Bilder
\usepackage{graphicx}
\makeatletter
\def\ScaleIfNeeded{%
	\ifdim\Gin@nat@width>\linewidth
	\linewidth
	\else
	\Gin@nat@width
	\fi
}
\makeatother
\usepackage{color} % Farben
\graphicspath{{images/}}
\DeclareGraphicsExtensions{.pdf,.png,.jpg} % bevorzuge pdf-Dateien
\usepackage{subfigure} % mehrere Abbildungen nebeneinander/übereinander
\newcommand{\subfigureautorefname}{\figurename} % um \autoref auch für subfigures benutzen
\usepackage[all]{hypcap} % Beim Klicken auf Links zum Bild und nicht zu Caption gehen

% Bildunterschrift
\setcapindent{0em} % kein Einrücken der Caption von Figures und Tabellen
\setcapwidth[c]{0.9\textwidth}
\setlength{\abovecaptionskip}{0.2cm} % Abstand der zwischen Bild- und Bildunterschrift

% Quellcode
\usepackage{listings} % für Formatierung in Quelltexten
\definecolor{grau}{gray}{0.25}
\definecolor{darkgreen}{RGB}{0,140,0}
\definecolor{darkred}{RGB}{140,0,0}
\lstset{
	language=Java,
	frame=lines,
	extendedchars=true,
	basicstyle=\tiny\ttfamily,
	%basicstyle=\footnotesize\ttfamily,
	tabsize=2,
	keywordstyle=\color{blue},
	commentstyle=\color{darkgreen},
	stringstyle=\color{darkred},
	numbers=left,
	numberstyle=\tiny,
	% für Zeilenumbruch
	breakautoindent  = true,
	breakindent      = 0em,
	breaklines       = true,
	postbreak        = ,
	prebreak         = \raisebox{-.8ex}[0ex][0ex]{\Righttorque},
}

% linksbuendige Fussnoten
\deffootnote{1.5em}{1em}{\makebox[1.5em][l]{\thefootnotemark}}
\typearea{14} % typearea am Schluss berechnen lassen, damit die Einstellungen oben beruecksichtigt werden
% für autoref von Gleichungen in itemize-Umgebungen
\makeatletter
\newcommand{\saved@equation}{}
\let\saved@equation\equation
\def\equation{\@hyper@itemfalse\saved@equation}
\makeatother 

\usepackage[justification=justified, singlelinecheck=false, labelfont={bf,small,sf}, font={small,sf}, aboveskip=0.5cm,belowskip=0em]{caption}

\usepackage[xindy]{glossaries}

% Abstand vor Chapter verkleinern
\renewcommand*{\chapterheadstartvskip}{\vspace*{-\topskip}} %weniger Abstand vor chapter 
\renewcommand*{\chapterheadendvskip}{\vspace{0.7cm}} %weniger Abstand nach chapter

% Neue Befehle
\newcommand{\todo}[1]{
      {\colorbox{red}{ TODO: #1 }}
}
\newcommand{\info}[1]{
      {\colorbox{blue}{ (INFO: #1)}}
}
\newcommand{\code}[1]{
      {\ttfamily{#1}}
}